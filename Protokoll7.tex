% Für Bindekorrektur als optionales Argument "BCORfaktormitmaßeinheit", dann
% sieht auch Option "twoside" vernünftig aus
% Näheres zu "scrartcl" bzw. "scrreprt" und "scrbook" siehe KOMA-Skript Doku
\documentclass[12pt,a4paper,titlepage,headinclude,bibtotoc]{scrartcl}


%---- Allgemeine Layout Einstellungen ------------------------------------------

% Für Kopf und Fußzeilen, siehe auch KOMA-Skript Doku
\usepackage[komastyle]{scrpage2}
\pagestyle{scrheadings}
\automark[section]{chapter}
\setheadsepline{0.5pt}[\color{black}]


%Einstellungen für Figuren- und Tabellenbeschriftungen
\setkomafont{captionlabel}{\sffamily\bfseries}
\setcapindent{0em}


%---- Weitere Pakete -----------------------------------------------------------
% Die Pakete sind alle in der TeX Live Distribution enthalten. Wichtige Adressen
% www.ctan.org, www.dante.de

% Sprachunterstützung
\usepackage[ngerman]{babel}

% Benutzung von Umlauten direkt im Text
% entweder "latin1" oder "utf8"
\usepackage[utf8]{inputenc}

% Pakete mit Mathesymbolen und zur Beseitigung von Schwächen der Mathe-Umgebung
\usepackage{latexsym,exscale,amssymb,amsmath}

% Weitere Symbole
\usepackage[nointegrals]{wasysym}
\usepackage{eurosym}

% Anderes Literaturverzeichnisformat
%\usepackage[square,sort&compress]{natbib}

% Für Farbe
\usepackage{color}

% Zur Graphikausgabe
%Beipiel: \includegraphics[width=\textwidth]{grafik.png}
\usepackage{graphicx}

% Text umfließt Graphiken und Tabellen
% Beispiel:
% \begin{wrapfigure}[Zeilenanzahl]{"l" oder "r"}{breite}
%   \centering
%   \includegraphics[width=...]{grafik}
%   \caption{Beschriftung} 
%   \label{fig:grafik}
% \end{wrapfigure}
\usepackage{wrapfig}

% Mehrere Abbildungen nebeneinander
% Beispiel:
% \begin{figure}[htb]
%   \centering
%   \subfigure[Beschriftung 1\label{fig:label1}]
%   {\includegraphics[width=0.49\textwidth]{grafik1}}
%   \hfill
%   \subfigure[Beschriftung 2\label{fig:label2}]
%   {\includegraphics[width=0.49\textwidth]{grafik2}}
%   \caption{Beschriftung allgemein}
%   \label{fig:label-gesamt}
% \end{figure}
\usepackage{subfigure}

% Caption neben Abbildung
% Beispiel:
% \sidecaptionvpos{figure}{"c" oder "t" oder "b"}
% \begin{SCfigure}[rel. Breite (normalerweise = 1)][hbt]
%   \centering
%   \includegraphics[width=0.5\textwidth]{grafik.png}
%   \caption{Beschreibung}
%   \label{fig:}
% \end{SCfigure}
\usepackage{sidecap}

% Befehl für "Entspricht"-Zeichen
\newcommand{\corresponds}{\ensuremath{\mathrel{\widehat{=}}}}

%Für chemische Formeln (von www.dante.de)
%% Anpassung an LaTeX(2e) von Bernd Raichle
\makeatletter
\DeclareRobustCommand{\chemical}[1]{%
  {\(\m@th
   \edef\resetfontdimens{\noexpand\)%
       \fontdimen16\textfont2=\the\fontdimen16\textfont2
       \fontdimen17\textfont2=\the\fontdimen17\textfont2\relax}%
   \fontdimen16\textfont2=2.7pt \fontdimen17\textfont2=2.7pt
   \mathrm{#1}%
   \resetfontdimens}}
\makeatother

%Si Einheiten
\usepackage{siunitx}

%c++ Code einbinden
\usepackage{listings}
\lstset{numbers=left, numberstyle=\tiny, numbersep=5pt}

\usepackage{caption}

%Zitate
%\usepackage[round]{natbib}
\usepackage{cite}

%keine Einrückung nach leerzeile
\parindent0pt

%Literaturverzeichnis
\usepackage{babelbib}
\selectbiblanguage{ngerman}

\begin{document}

\begin{titlepage}
\centering
\textsc{\Large Anfängerpraktikum der Fakultät für
  Physik,\\[1.5ex] Universität Göttingen}

\vspace*{4.2cm}

\rule{\textwidth}{1pt}\\[0.5cm]
{\huge \bfseries
  Adiabatenexponent\\[1.5ex]
  Protokoll:}\\[0.5cm]
\rule{\textwidth}{1pt}

\vspace*{3.0cm}

\begin{Large}
\begin{tabular}{ll}
Praktikant:
%	&  Skrollan Detzler\\
 	&  Felix Kurtz\\
% 	&  Michael Lohmann\\
	&  Kevin Lüdemann\\

  E-Mail: 
%	&  skrollan.detzler@stud.uni-goettingen.de\\
	&  felix.kurtz@stud.uni-goettingen.de\\
%	& m.lohmann@stud.uni-goettingen.de\\	
	&  kevin.luedemann@stud.uni-goettingen.de\\

 Betreuer: & Martin Ochmann\\
 Versuchsdatum: & 16.06.2014\\
\end{tabular}
\end{Large}

\vspace*{0.8cm}

\begin{Large}
\fbox{
  \begin{minipage}[t][2.5cm][t]{6cm} 
    Testat:
  \end{minipage}
}
\end{Large}

\end{titlepage}

\tableofcontents

\newpage

\section{Einleitung}
\label{sec:einleitung}
Der Adiabatenexponent ist ein Ausdruck, der durch die Anzahl der Freiheitsgrade eines Gasen bestimmt wird. 
Alternativ kann dieser auch über die spezifische Wärme eines Gases bestimmt werden.
Wir wollen in diesem Versuch auf zwei verschiedene Arten diesen Adiabatenexponenten $\kappa$ bestimmen.
Der erste Versuchsaufbau ist der Aufbau nach Rüchardt und der zweite ist nach Clement-Desormes.

\section{Theorie}
\label{sec:theorie}

\subsection{Ideal Gas}
Als Ideal Gas wird ein Gas bezeichnet, dass als einatomig angesehen wird.
Zum vereinfach der Rechnungen wird die Annahme gemacht, dass dieses Gas nur von Punktteilchen gefüllt ist.
Dies vereinfacht die Darstellung von Gesetzen und ermöglich es die Ideal Gasgleichung auf zu stellen, dessen Verhältnis von Durck und Volumen nur von der Zahl der Teilchen und deren Temperatur abhängig ist.
\begin{align}
	pV=Nk_BT=nRT\label{eq:ideal}
\end{align}
Hierbei ist die Anzahl der Teilchen N und die Boltzman konstante k$_B$ zusammengefasst zu R=N$_A$k$_B$, wobei N$_A$ die Advogardo konstante ist n die Stoffmenge.

\subsection{Zustandsänderung in Gasen}
Die Ideal Gas Gleichung \eqref{eq:ideal} gilt für ein Ideal Gas immer, dennoch kann es sich auf verschiedene Zusatndsänderungen verschieden verhalten.
Bei z.B. bleibt der Druck p konstannt, so ist das Volumen V proportional zur Temperatur T.
Diese Zustandsänderung wird als Isobar bezeichnet.
Belibt hingegen das Volumen konstant, so spricht man von einer Isochoren Zusatndsänderung.
Die hier interessantere Änderung ist aber die adiabatische Änderung.
Hierbei bleibt die Temperatur konstant und es ändern sich Druck und Volumen, doch das Verhältnis zwischen den beiden bleibt gleich.
Dies ist im allgemeinen sehr schwierig zu relisieren, doch es gibt mitlerweile recht gute Isolierungen, oder der Prozess, der adiabatisch ablaufen soll, wird sehr schnell ausgeführt.

\subsection{Herleitung der Poisson-Gleichung}


\subsection{Freiheitsgrade}
Die Spezifische Wärme eines Gases ist eine Konstante und aus dieser lässt sich mithilfe der Freiheitsgrade und der Konstanten R, wie oben beschriben, die Konstante $\kappa$ erstellt.
Die Zusammenhange ergeben sich aus diesen beiden Formel %Quelle hinzufuegen
\begin{align}
	c_v=\frac{f}{2}R\\
	c_p=\left(\frac{f}{2}+1\right)R
\end{align}
Teilt man jetzt $c_p$ durch $c_v$, so erhällt man den Adiabatenexponenten $\kappa$, welcher durch konstanten festgelegt ist.
\begin{align}
	\kappa=\frac{c_p}{c_v}=\frac{f+2}{f}
\end{align}

\subsection{Adiabatenexponent aus Versuchsaufbau}

\subsection{Nach Rüchardt}

\subsection{Nach Clement-Desormes}

\cite[S.$\infty$]{prakti}

\section{Durchführung}
\label{sec:durchfuehrung}

\subsection{Adiabatenexponent nach Rüchardt}
Der Versuch besteht aus einer Glasskugel, die nach oben hin eine lange dünne Zylindrische Öffnung hat.
Der Aufbau ist in der Graphik %ref auf Graphik
zu sehen.
Zum Einlassen des Gases existiert eine Zuleitung, welche über eine Druckkopplungsventiel mit drei verschiedenen Gasquellen an der Wand verbuunden werden kann.
Am langen Rohr oberhalb der Kugel ist eine Lichtschranke befestigt und es gibt einen kleinen Schlitz unterhalb der Lichtschranke, um das Gas entweichen lassen zu könnne.
Im Rohr ist ein ebenfalls Zylindrischer Körper, der eng mit der Glasswand des Rohres Abschließt, so dass fast kein Gas vorbei kommt.
Es ist sich mit der ausliegenden Bedienungsanleitung der Lichtschranke vertraut zu machen.\\
Bevor der Versuch gestartet werden kann, muss die Zukleitung mit einem der drei Gasqeullen verbunden werden und das Regulierungsventiel geöffnet werden.
Damit die Kugel nur noch mit dem gewünschten Gas gefüllt ist, muss die Kugel vor begin des Versuches mindestens 3 Minuten mit dem Gas durchgespüklt werden.
Hierzu wird das Regulierungsventiel und das Entlüftungsventil aufgedreht.\\
Ist dieser Vorgang abgeschlossen, so wird das Entlüftungsventil wieder geschlossen und anschließend das Regulierungsventiel so eingestellt, dass sich der kleine Zylindrische Körper in einer gleichmäßigen Schwingung um die Lichtschranke befindet.
Es werden jetzt, ohne die Gasregulierung stark zu ändern Messungen von 10 mal einer Schwingung und jeweils 3 mal von 10, 20, 50 und 100 Schwingungen durchgeführt.
Hierzu muss nur die Lichtschranke bedient zu werden.
Dies wird jetzt für die Gase Luft, C0$_2$ und Argon durchgeführt.
Man beginnt jeweils wieder mit dem durchspülen der Kugel.
Es ist schließlich noch nötig die Masse des kleinen Körpers, der Rohrinnendurchmesser und das Volumen von Kolben und Rohr zu nozieren.

\subsection{Adiabatenexponent nach Clement-Desormes}
Der Versuchsaufbau ist ebenfalls in der Graphik %ref auf Graphik
von oben zu sehen.
Dieser Aufbau besteht aus einem großen Glaszylinder und einem Manometer.
Zu dem gibt es noch ein Entlüftungs und Verschlussventil und einen Blasebalg zum erzeugen des Drucks.\\
Zu Begin des Versuches ist das Verschlussventil zu öffnen und mit dem Blasebalg ein höherer Druck zu erzeugen.
Ist der gewünschte Druck erreicht, so verschließt man das Verschlussventil und führe einen Temperaturausgleich durch.
Für diesen, wird eine Weile lang gewartet, bis sich die Temperatur um Zylinder mit der, der Umwelt ausgeglichen hat.
Anschließend notiert man sich die Höhendifferenz auf dem Manometer.\\
Es werden jetzt jeh 3 Messungen für verschiedene Öffnungszeiten des Entlüftungsventils gemacht.
Hierzu wird dieses für ca. 0.1s, 1s und 5s geöffnet.
Nach dem kurzen öffnen und schließen ist wieder ein Temperaturausgleich durch zu führen.
Danach wird dann wieder die Höhendifferenz aufgeschriben.
Es wird wieder, wie oben geschrieben der Druck erhöht und mit der Messreihe vortgefahren, bis alle Messungen durchgeführt sind.
Wichtig ist hierbei stets den Temperatturausgleich zu machen und vor jeder neuen Messung neuen Druck auf zu bauen.

\section{Auswertung}
\label{sec:auswertung}
\subsection{Rüchardt}

\begin{table}[!hbt]
	\centering
	\begin{tabular}{|c|c|}
		\hline
		Größe & Wert\\
		\hline
		\hline
		Masse & $m = 8.432~\si{\gram}$\\
		Durchmesser & $d = 11.93~\si{\milli\meter}$\\
		Volumen & $V = 2225~\si{cm^3}$\\
		\hline
		Luftdruck & $b = (1015.7 \pm 0.1)~\si{hPa}$\\ 		
		\hline
		Dichte von Luft & $\rho_L = 1.2~\si{kg/m^3}$\\
		Erdbeschleunigung & $g = 9.81~\si{m/s^2}$\\		
		\hline
	\end{tabular}
\end{table}

\begin{table}[!hbt]
	\centering
	\begin{tabular}{|c|c|}
		\hline
		Gas & Amplitude $l$ [cm]\\
		\hline
		\hline
		CO$_2$& $19.5\pm0.5$\\
		Argon & $12.5\pm0.5$\\
		Luft & $17.5\pm0.5$\\
		\hline
	\end{tabular}
\end{table}

$$ A = \pi\frac{d^2}{4}$$

\begin{align}
	m_{\text{eff}}&= m + \rho_L \cdot A \cdot l\\
	\sigma_{m_\text{eff}}&=\rho_L \cdot A \cdot \sigma_l
\end{align}

\begin{align}
	p &= b + m_{\text{eff}} \cdot \frac{g}{A}\\
	\sigma_p &= \sqrt{\sigma_b^2+\sigma_{m_{\text{eff}}}^2 \cdot \left(\frac{g}{A}\right)^2}\\
	&=\sqrt{\sigma_b^2+\left(\rho_L \cdot g\right)^2 \cdot \sigma_l^2 }
\end{align}

\begin{table}[!hbt]
	\centering
	\begin{tabular}{|c|c|c|}
		\hline
		Gas & $m_{\text{eff}}$ [g] & $p$ [hPa] \\
		\hline
		\hline
		CO$_2$ & $8.4582 \pm 0.0007$ & $1023.12 \pm 0.10$ \\
		Argon & $8.4488 \pm 0.0007$ & $1023.11 \pm 0.10$ \\
		Luft & $8.4555 \pm 0.0007$ & $1023.12 \pm 0.10$ \\
		\hline
	\end{tabular}
\end{table}


\begin{align}
	\kappa&=\frac{64 \cdot m_{\text{eff}}\cdot V }{T^{2} \cdot p \cdot d^{4}}\\
	\sigma_{\kappa}&=\frac{64 ~ V}{T^{3} ~ d^{4} ~ p^{2}} \cdot \sqrt{\left(T ~ m_{\text{eff}}\right)^2 \cdot \sigma_{p}^{2} + \left(T ~ p\right)^2 \cdot \sigma_{m_{\text{eff}}}^{2} + \left(2~m_{\text{eff}}~p\right)^{2} \cdot \sigma_{T}^{2}}
\end{align}

\begin{align}
	f&=\frac{2}{\kappa - 1}\\
	\sigma_{f}&=\frac{2 \cdot \sigma_{\kappa}}{\left(\kappa - 1\right)^{2}}
\end{align}

\begin{table}[!hbt]
	\centering
	\begin{tabular}{|c|c|c|}
		\hline
		Gas & $\kappa$ & $f$\\
		\hline
		\hline		
		CO$_2$ & $1.3037 \pm 0.0005$ & $6.585 \pm 0.011$ \\
		Argon & $1.5944 \pm 0.0010$ & $3.365 \pm 0.006$ \\
		Luft & $1.4051 \pm 0.0008$ & $4.937 \pm 0.009$ \\		
		\hline
	\end{tabular}
\end{table}


\subsection{Clement-Desormes}

\begin{align}
	\kappa&=\frac{\Delta h_{1}}{\Delta h_{1} - \Delta h_{2}}\\
	\sigma_{\kappa}&=\frac{1}{\left(\Delta h_{1} - \Delta h_{2}\right)^{2}} \cdot \sqrt{\Delta h_{1}^{2} \cdot \sigma_{\Delta h_2}^{2} + \Delta h_{2}^{2} \cdot \sigma_{\Delta h_1}^{2}}
\end{align}

\begin{table}[!htb]
	\centering
	\begin{tabular}{|c|c|}
		\hline
		Öffnungszeit [s] & $\kappa$\\
		\hline
		$ 0.1 $ & $ 1.205 \pm 0.022 $ \\
		$ 1.0 $ & $ 1.227 \pm 0.022 $ \\
		$ 5.0 $ & $ 1.177 \pm 0.018 $ \\
		\hline
	\end{tabular}
	\caption{gewichtete Mittelwerte von $\kappa$ \\ für die verschiedenen Öffnungszeiten}
\end{table}

\subsection{Mittelwert für $\kappa_\text{Luft}$ aus beiden Messungen}
\begin{align}
	\kappa_\text{Luft}=1.4042 \pm 0.0008
\end{align}

\section{Diskussion}
\label{sec:diskussion}

\section{Anhang}
\begin{table}[!htb]
	\centering
	\begin{tabular}{|c|c|c|c|}
		\hline
		Gas & Schwingungen & Periodendauer [ms] &$\kappa$\\
		\hline
		\hline
		CO$_2$
		& 1 & $663.9 \pm 1.0$ & $1.319 \pm 0.004$ \\
		& 10 & $666.20 \pm 0.17$ & $1.3094 \pm 0.0007$ \\
		& 20 & $667.7 \pm 0.4$ & $1.3035 \pm 0.0015$ \\
		& 50 & $670.5 \pm 0.4$ & $1.2925 \pm 0.0015$ \\
		& 100 & $672.0 \pm 0.4$ & $1.2870 \pm 0.0014$ \\
		\hline
		Argon
		& 1 & $601.6 \pm 1.0$ & $1.604 \pm 0.005$ \\
		& 10 & $602.80 \pm 0.25$ & $1.5976 \pm 0.0013$ \\
		& 20 & $604.07 \pm 0.31$ & $1.5909 \pm 0.0016$ \\
		& 50 & $606.8 \pm 1.1$ & $1.577 \pm 0.006$ \\
		& 100 & $615.0 \pm 3.1$ & $1.535 \pm 0.016$ \\
		\hline		
		Luft
		& 1 & $639.3 \pm 1.0$ & $1.422 \pm 0.005$ \\
		& 10 & $641.03 \pm 0.29$ & $1.4138 \pm 0.0013$ \\
		& 20 & $642.5 \pm 0.4$ & $1.4073 \pm 0.0016$ \\
		& 50 & $644.5 \pm 0.5$ & $1.3988 \pm 0.0023$ \\
		& 100 & $646.4 \pm 0.4$ & $1.3906 \pm 0.0016$ \\
		\hline
	\end{tabular}
\end{table}

\bibliography{literatur}
\bibliographystyle{babalpha}

\end{document}
